% !TEX encoding = UTF-8 Unicode
%!TEX root = main.tex
% !TEX spellcheck = en-US
%%=========================================

\newpage
\phantomsection
\section*{Preface}
\addcontentsline{toc}{section}{Preface}


This master thesis presents the results of the project work, affiliated with the course TTK4900
from Department of Engineering Cybertnertics (ITK) at the Norwegian University of Science
and Technology (NTNU). The project was carried out during the spring semester in 2017, starting in January and ending in June.

The majority of the work has gone into the implementation of the runtime scheduler, as well as debugging strange heisenbugs. As an afterthought, who would have guessed multi\hyp{}core systems are easy to debug (spoiler: they are not). Hopefully, this thesis reflects the amount of work gone into the project development.

The reader is expected to basic knowledge within programming, especially concurrent programming. Having experience with a modern dialect of C++, at least C++11 or later, helps as well. 

I would like to thank both of my supervisors, Sverre Hendseth and Øyvind Teig, for great feedback and knowledge regarding concurrent system design and paradigms, and for enjoyable and interesting academic evenings at Solsiden.


\begin{flushright}
\texttt{Edvard Severin Pettersen}\\
\includegraphics[width=0.3\linewidth,right]{fig/signature}
Trondheim, 2017-06-04
\end{flushright}

\afterpage{\blankpage}
