% !TEX encoding = UTF-8 Unicode
%!TEX root = main.tex
% !TEX spellcheck = en-US
%%=========================================


%%%%%%%%%%%%%%%%%%%%%%%%%%%%%%%%%%%%%%%%%%%%%%%%%%%%%%%%%%%%%%%%%%%%%%%%%%%%%%%%
\chapter{Implementation}
\label{ch:implementation}


The library is written in C++, with standard C++14 dialect. The reader is expected to have a fair understanding of C++, and being familiar with standard C++11 dialect or newer is recommended. Detailed explanations of the C++ programming language is not presented here. Refer to any C++ reference (e.g. \citet{stroustrup2013c++}) for more details.


\FloatBarrier
%%%%%%%%%%%%%%%%%%%%%%%%%%%%%%%%%%%%%%%%%%%%%%%%%%%%%%%%%%%%%%%%%%%%%%%%%%%%%%%%
\section{Data Structures}


A set of data structures is commonly used by the runtime system. Apart from the C++ Standard Template Library (STL), the most notable data structures are \textit{intrusive containers}, \textit{concurrent queues}, and \textit{mutual exclusion locks}.

Intrusive containers, just as any other containers, stores some kind of data in some sort of way. The difference is how the container stores the necessary data used to organize the data. A non\hyp{}intrusive container is responsible for storing the necessary data, while for an intrusive container the elements are responsible for storing the necessary data. In other words, the element becomes ``aware'' of being a part of the intrusive container. Usually, intrusive containers are implemented with the elements having \textit{hooks} as data members. These hooks contains all the necessary data used by the intrusive container to store the elements. Intrusive containers offers better performance compared to non\hyp{}intrusive containers, as they minimize memory allocations and better memory locality.

Concurrent queues are queues which are safe to use concurrently, often denoted as \textit{thread safe}. The most common approach is taking a non\hyp{}thread safe queue and enforcing mutual exclusion around the critical regions. This approach however is not desirable, as it has very low throughput in multiprogrammed programs. Plenty of research \citep[e.g.][]{chase2005dynamic,le2013correct} has been devoted to creating non\hyp{}blocking queues (see \cref{sec:nonblocking_algorithms}) both available and efficient. Concurrent queues often differentiate between single or multiple producers and consumers. Producers are processes which insert elements into the queue, and consumers are processes which remove elements from the queue. The runtime system uses the variants \textit{single\hyp{}producer\hyp{}multiple\hyp{}consumer} (SPMC) queues and \textit{multiple\hyp{}producer\hyp{}single\hyp{}consumer} (MPSC) queues.

Creating a complete non\hyp{}blocking system is most of the times impossible for a multiprogrammed programs. Sometimes resorting to mutual exclusion in critical regions is unavoidable. Different types of locks is suitable for different situations. Whether the lock is often contested, meaning multiple kernel\hyp{}threads are trying to acquire the lock simultaneously, and if the lock is held for a longer period of time or not, will affect the performance. \Citet[page 196--199]{brown2007c++csp2} performs a case study on different mutexes, describing various mutex algorithms and provides a benchmark and analysis of their performance. The conclusion from the case study is that for low contested, short\hyp{}term held mutexes, spinlocks yields best performance regarding low latency. 

For multiprocessor architectures, the \textit{test\hyp{}and\hyp{}test\hyp{}and\hyp{}set} (TTAS) spinlock is generally favorable as it causes less memory contention than the standard spinlock. Instead of constantly trying to test\hyp{}and\hyp{}set the lock, it waits until the lock appears free. Different variants of the TTAS spinlock includes constant/exponential backoff during contention and cache friendly atomic operations.

\section{Lightweight Process Implementation}


\section{Scheduler Implementation}


\section{Timer Implementation}

\section{Channel Implementation}


\section{Alting Implementation}

