% !TEX encoding = UTF-8 Unicode
%!TEX root = main.tex
% !TEX spellcheck = en-US
%%=========================================


%%%%%%%%%%%%%%%%%%%%%%%%%%%%%%%%%%%%%%%%%%%%%%%%%%%%%%%%%%%%%%%%%%%%%%%%%%%%%%%%
\chapter{Examples of Usage}
\label{ch:examples_usage}

This chapter introduces how \Proxc{} is used in \Cpp{} programs. Only the basics of \Proxc{} are presented, introducing all necessary features for understanding the library. First of all, \Proxc{} must be installed on the work environment. \Proxc{} is available for free as in speech on GitHub \citep{pettersen2017proxcgithub}. Following the install guide on the readme should be sufficient. For more advanced examples, check out the examples in the \texttt{examples/} folder on the GitHub project.

All library related types and methods reside in the \lstinline[style={CustomC++}]|proxc| namespace, which will be omitted in all code examples. It is given the header file \lstinline[style={CustomC++}]|#include <proxc.hpp>| is included in all code examples shown in this chapter, as well as in all \Cpp{} files programming with \Proxc. The linker flag \lstinline[style={CustomC++}]|-lproxc| must also be passed during compilation.

Programs using \Proxc{} does not need to initialize or cleanup the library. This is done automatically by the library. The first call to the library which invokes the runtime system will initialize the library, and the cleanup procedure is invoked when the program exits. No cleanup will be invoked unless the library has been initialized. 


%%%%%%%%%%%%%%%%%%%%%%%%%%%%%%%%%%%%%%%%
\section{Processes}


At the core of \Proxc{} are lightweight processes. These processes are a separate scope of execution which can run concurrently with the rest of the program. In code, these processes are represented by a function and its corresponding arguments. The main function of any \Cpp{} program is also implicitly a process. This function must have a return type of \lstinline[style={CustomC++}]|void|, but can have any type and number of arguments. \Cref{lst:code_example_process_func} shows different function prototypes where some qualify running as a process and some do not.

\begin{lstfloat}
\begin{lstlisting}[caption={Different function prototypes which do and do not qualify as a process.}, label={lst:code_example_process_func}, style={CustomC++}, xleftmargin={2em}]
void good_func1();                // ok
void good_func2(std::string msg); // also ok
int         bad_func1();      // not ok, return type not void
std::string bad_func2(int y); // also not ok
\end{lstlisting}
\end{lstfloat}

Processes are stored as process objects of the type \lstinline[style={CustomC++}]|Process|. These process objects can be created and stored freely in any container. A process object can be created explicitly with the object constructor, which takes a function pointer and its corresponding arguments. Alternatively, a process object can be created implicitly with the library method \lstinline[style={CustomC++}]|proc|. \Cref{lst:process_creation} illustrates the difference.

\begin{lstfloat}[h!]
\begin{lstlisting}[caption={Process creation.}, label={lst:process_creation}, style={CustomC++}, xleftmargin={2em}]
Process my_process{&my_func, arg1, arg2};
auto other_process = proc(&other_func, other_arg);
\end{lstlisting}
\end{lstfloat}

Process creation also has perfect forwarding of arguments, meaning any type of arguments being expensive to copy or being non\hyp{}copyable can correctly be moved into processes.

A dynamic number of processes can also be created, which requires to either explicitly allocate each process before hand in any container or to generate them on the fly. The library method \lstinline[style={CustomC++}]|proc_for| takes either a pair of input iterators to any container of processes, or a pair of integers which defines the integer range $[\text{start};\text{end}\rangle$ and a function pointer which takes an integer as an argument. The method returns a process which runs the dynamic numbers of processes in parallel. See \cref{lst:dynamic_number_process_creation} for reference. Calling the \lstinline[style={CustomC++}]|proc_for| method only creates the parallel process, and does not run them.

\begin{lstfloat}[h!]
\begin{lstlisting}[caption={Creating a dynamic number of processes.}, label={lst:dynamic_number_process_creation}, style={CustomC++}, xleftmargin={2em}]
std::vector<Process> procs;
for (int i = 0; i < N; ++i) 
    procs.emplace_back(&some_func, some_data[i]);
auto total_procs = proc_for(procs.begin(), procs.end());
// or
auto total_procs = proc_for(0, N,
    [&some_data](int i){ some_func(some_data[i]); });
\end{lstlisting}
\end{lstfloat}

Creating a process is not useful if it cannot be executed in parallel with the rest of the system. The library function \lstinline[style={CustomC++}]|parallel| takes one\hyp{}or\hyp{}more processes and runs them in parallel, following the fork\hyp{}join model. The calling process will suspend until all parallel processes have terminated. See \cref{lst:example_parallel_statement} for reference.

\begin{lstfloat}
\begin{lstlisting}[caption={Example of the parallel statement.}, label={lst:example_parallel_statement}, style={CustomC++}, xleftmargin={2em}]
std::vector<Process> procs;
// ...
parallel(
    proc(&my_func, 42),
    proc([](){ /* lambda function */ }),
    proc_for(procs.begin(), procs.end()),
    proc_for(0, N, &calculate_func)
);
\end{lstlisting}
\end{lstfloat}

Process objects are non\hyp{}copyable but movable, meaning the ownership of process objects must be moved between scopes. A process object is of one\hyp{}time use. After a process has executed and terminated, the process object cannot be run again. The process object must be constructed once more to be executed again. 

In any process context a set of operations are available for the current running process, accessible in the \lstinline[style={CustomC++}]|this_proc| namespace. The operations includes getting the current process id, yielding, and suspending for a given duration or until a time point. The suspension operations supports time types from the standard library \lstinline[style={CustomC++}]|std::chrono|. See \cref{lst:per_process_operations} for reference.

\begin{lstfloat}
\begin{lstlisting}[caption={Per process operations for the current executing process.}, label={lst:per_process_operations}, style={CustomC++}, xleftmargin={2em}]
auto id = this_proc::get_id();
this_proc::yield();
this_proc::delay_for(/* duration */);
this_proc::delay_until(/* time point */);
\end{lstlisting}
\end{lstfloat}


%%%%%%%%%%%%%%%%%%%%%%%%%%%%%%%%%%%%%%%%
\section{Timers}
\label{sec:timer_example}


Three types of timers are available in \Proxc{}: egg, repeat and date timers. Timers allow for soft real\hyp{}time requirements on certain operations. All timers have support for time period and duration declarations using the standard time library \lstinline[style={CustomC++}]|std::chrono|.

All timer objects reside in the \lstinline[style={CustomC++}]|timer| namespace. Timer objects are constructed with a supplied appropriate duration or time point, and are both copyable and movable. See \cref{lst:creating_timers} for reference. 

\begin{lstfloat}
\begin{lstlisting}[caption={Constructing different timers.}, label={lst:creating_timers}, style={CustomC++}, xleftmargin={2em}]
timer::Egg    egg{/* duration */};
timer::Repeat rep{/* duration */};
timer::Date   date{/* time point */};
\end{lstlisting}
\end{lstfloat}

Egg timers are used for relative timeout periods. Just as an egg timer in real life, the timer starts when the operation starts, and expires when the period has passed. The egg timer can be reused after expiration, as the countdown is reset every time it is supplied for an operation. In a sense, the timer does not ``survive'' between multiple operations, as it resets.

Repeat timers are used for periodic timeout periods. Given a time duration, the repeat timer will expire every period equal to the time duration. When supplied with an operation, the repeat timer does not reset. Only when it expires does it reset, basically setting the next timeout point to the next period forward in time. Repeat timers do ``survive'' between multiple operations, as time timer only resets when it expires. 

Date timers are used for absolute timeout periods. Given a time point, the date timer will expire when current time point has surpassed the given time point. The date timer does ``survive'' between multiple operations, however, always stays expired after a timeout. 


%%%%%%%%%%%%%%%%%%%%%%%%%%%%%%%%%%%%%%%%
\section{Channels}


Processes use message\hyp{}passing to communicate, which is realized through channels. The philosophy of CSP\hyp{}based concurrency is processes should not share memory directly, as this is a potent problem often leading to race conditions. Rather, the processes should share memory by communicating, i.e. message\hyp{}passing with channels.

Channels in \Proxc{} exist only in one flavour: synchronous and unbuffered, unidirectional, one\hyp{}to\hyp{}one and type safe. Channel objects are of the type \lstinline[style={CustomC++}]|Chan<T>|, and takes no additional arguments in its constructor. Channel objects are non\hyp{}copyble but movable.

A channel object is a named tuple, containing the two channel end objects \lstinline[style={CustomC++}]|Chan<T>::Tx| and \lstinline[style={CustomC++}]|Chan<T>::Rx|, The channel ends \lstinline[style={CustomC++}]|Tx| and \lstinline[style={CustomC++}]|Rx| can respectively send and receive on the channel. Channel end objects are also non\hyp{}copyable but movable.

Channel ends can be access directly from the channel object via the class methods \lstinline[style={CustomC++}]|ref_tx| and \lstinline[style={CustomC++}]|ref_rx|, which returns a reference to the channel end objects. Channel end objects can also be moved from the channel object with the class methods \lstinline[style={CustomC++}]|move_tx| and \lstinline[style={CustomC++}]|move_rx|. See \cref{lst:channel_creation} for reference.

\begin{lstfloat}
\begin{lstlisting}[caption={Channel creation and channel end access.}, label={lst:channel_creation}, style={CustomC++}, xleftmargin={2em}]
Chan<std::string> channel;
/* channel end access */
channel.ref_tx();  /* and */ channel.ref_rx();
/* channel end move */
channel.move_tx(); /* and */ channel.move_rx();
\end{lstlisting}
\end{lstfloat}

Given a channel of type \lstinline[style={CustomC++}]|Chan<T>|, the channel end objects \lstinline[style={CustomC++}]|Tx| and \lstinline[style={CustomC++}]|Rx| can send data of the type \lstinline[style={CustomC++}]|T| over the channel. Channel transmissions has perfect forwarding of data.

The channel end object \lstinline[style={CustomC++}]|Tx| can either send data with the class method \lstinline[style={CustomC++}]|send| or with the overloaded \lstinline[style={CustomC++}]|operator<<|. The class method returns the explicit result of the operation, while the overloaded operator only returns a boolean of whether the operation was a success or not. See \cref{lst:sending_channel_end} for reference.

\begin{lstfloat}
\begin{lstlisting}[caption={Sending on the Tx channel end object.}, label={lst:sending_channel_end}, style={CustomC++}, xleftmargin={2em}]
Chan<long> channel;
auto tx = channel.move_tx();
auto op_result = tx.send(some_data); 
/* or */
bool success = tx << some_data;
\end{lstlisting}
\end{lstfloat}

The channel end object \lstinline[style={CustomC++}]|Rx| can either receive data with the class method \lstinline[style={CustomC++}]|recv| or with the overloaded \lstinline[style={CustomC++}]|operator>>|. An important difference from the sending operations is the receive data must be preallocated before calling the receive methods. However, the return types resembles the return types of sending. See \cref{lst:receiving_channel_end} for reference.

\begin{lstfloat}
\begin{lstlisting}[caption={Receiving on the Rx channel end object.}, label={lst:receiving_channel_end}, style={CustomC++}, xleftmargin={2em}]
Chan<long> channel;
auto rx = channel.move_rx();
long some_data; // received data must be preallocated!
auto op_result = rx.recv(some_data); 
/* or */
bool success = tx >> some_data;
\end{lstlisting}
\end{lstfloat}

Both channel end objects overload \lstinline[style={CustomC++}]|operator()| for inline sending and receiving. However, inline channel operations does not return any indications whether the channel operation was a success or not. See \cref{lst:inline_channel_operations} for reference.

\begin{lstfloat}
\begin{lstlisting}[caption={Inline channel operations.}, label={lst:inline_channel_operations}, style={CustomC++}, xleftmargin={2em}]
Chan<T>::Tx tx; Chan<T>::Rx rx;
tx(data); // returns void
rx();     // returns T
\end{lstlisting}
\end{lstfloat}

A channel has a notion of being open or closed. A channel always starts as being open. When a channel is closed, no more channel transmission can be completed. A closed channel forever remains closed until destroyed. A channel becomes closed either by a channel end explicitly closing the channel with the class method \lstinline[style={CustomC++}]|close|, or a channel end object goes out of scope and is deallocated. A channel end can test whether the channel is closed or not with the class method \lstinline[style={CustomC++}]|is_closed|, which returns true for a closed channel and false otherwise.

The channel end object \lstinline[style={CustomC++}]|Rx| supports for range\hyp{}based for loops, retrieving data as long as the channel is open. When the channel becomes closed, the for loop automatically breaks. See \cref{lst:range_based_for_loops_channel} for reference.

\begin{lstfloat}
\begin{lstlisting}[caption={Range\hyp{}based for loops with the receving channel end.}, label={lst:range_based_for_loops_channel}, style={CustomC++}, xleftmargin={2em}]
Chan<T>::Rx rx;
for (auto data : Rx) {
    /* do some work */
}
/* here, channel is closed */
\end{lstlisting}
\end{lstfloat}

A receiving channel end can pipe its input into a sending channel end, meaning all data received on the receiving channel end is forwarded as the output on a sending channel end. The overloaded \lstinline[style={CustomC++}]|operator>>| is used to pipe data between a receiving and sending channel end, which returns a boolean indicating success or not. A successful piping is only when data is successfully received and sent. If either operation fails, the piping operation fails as well. See \cref{lst:piping_channel_end} for reference.


\begin{lstfloat}
\begin{lstlisting}[caption={Piping from a receiving to sending channel end object.}, label={lst:piping_channel_end}, style={CustomC++}, xleftmargin={2em}]
Chan<T>::Tx tx; Chan<T>::Rx rx;
while (rx >> tx) { /* piping is successful */ }
/* piping failed */
\end{lstlisting}
\end{lstfloat}

Lastly, a channel send or receive operation can be timed. A time duration or time point from the standard library \lstinline[style={CustomC++}]|std::chrono| can be supplied with a channel operation. The channel operation will try to complete within the specified time, and returns either if the operation completed or timed out. The return value of the timed operation specifies the reason for why the reason channel operation returned.


%%%%%%%%%%%%%%%%%%%%%%%%%%%%%%%%%%%%%%%%
\section{Alting}


If a process were to wait on multiple channel ends simultaneously, normal channel operations would not suffice. The alting construct however allows a process to simultaneously wait on multiple channel operations and complete one when one becomes ready. The channel operations, together with timeouts and skips, makes up what is called alting alternatives.

The four alternatives are as follows: channel send and receive takes an appropriate channel end and data, and performs the given channel operation. A timeout specifies a timer and wait until that timer has expired. A skip is always ready, much like the default keyword in a switch construct. 

The alting procedure allows a process to synchronously wait for multiple alternatives. The alting procedure block the calling process until one of the alternatives can complete, and completes that alternative. A corresponding closure, which can be anything callable, is then executed if present. When the alting procedure finishes, including the executed closure has returned, the process resumes.

An alting procedure consists of zero\hyp{}or\hyp{}more alternatives, where the alting procedure waits for these alternatives to become ``ready''. When one\hyp{}or\hyp{}more alternatives are ready, the alting procedure chooses one alternative and executes a corresponding closure. Alternatives can be guarded by a boolean condition, which enables or disables an alternative for the alting procedure depending on the condition.

An alting procedure starts with creating an alting object, which has the type \lstinline[style={CustomC++}]|Alt|. This alting object can create alternatives by function chaining multiple alternative methods, and lastly, call a selection method which consumes the alting object and resolves the alting procedure. See \cref{lst:alting_procedure} for reference.

\begin{lstfloat}
\begin{lstlisting}[caption={Example of the alting construct.}, label={lst:alting_procedure}, style={CustomC++}, xleftmargin={2em}]
Alt()
    .send(tx, 42)
    .recv_if(cond1, rx)
    .timeout(timer, &timeout_func)
    .skip_if(cond2, [](){
        /* skip lambda */
    })
    .select();
\end{lstlisting}
\end{lstfloat}

The four alternative types channel send, channel receive, timeout and skip has the corresponding alternative methods \lstinline[style={CustomC++}]|send|, \lstinline[style={CustomC++}]|recv|, \lstinline[style={CustomC++}]|timeout| and \lstinline[style={CustomC++}]|skip|, respectively. 

Each alternative method can prepend a guard with a boolean condition by calling the altered alternative method, which has an appended \lstinline[style={CustomC++}]|_if| to the method name, e.g. \lstinline[style={CustomC++}]|send_if|. All alternative methods, both with and without a guard, can have an optional closure. See \cref{lst:alting_procedure} for reference.

A dynamic number of channel send and receive alternatives can be generated. The class methods \lstinline[style={CustomC++}]|send_for| and \lstinline[style={CustomC++}]|recv_for| takes a pair of iterators to any container with the appropriate channel end object, and generates a corresponding alternative for each channel end. The class methods can also be guarded, following the same naming scheme as the single alternative methods. 

The timer alternatives takes on of the three timer types presented in \cref{sec:timer_example}. 

