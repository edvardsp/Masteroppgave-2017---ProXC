% !TEX encoding = UTF-8 Unicode
%!TEX root = main.tex
% !TEX spellcheck = en-US
%%=========================================


%%%%%%%%%%%%%%%%%%%%%%%%%%%%%%%%%%%%%%%%%%%%%%%%%%%%%%%%%%%%%%%%%%%%%%%%%%%%%%%%
\chapter{Conclusion}
\label{ch:conclusion}

Concurrent programming has long existed before multi\hyp{}core architectures entered the mainstream market. With the ever increasing use of multi\hyp{}core architectures, the demand for software which are scalable and fully utilize said processors are becoming more evident, and concurrent programming is the tool to realize such software.

For many programmers, it is hard to write correct concurrent systems with concurrent programming, and especially fail to utilize the available multiple processor cores. Communicating Sequential Processes (CSP) is a formal language for describing concurrent systems, which provides a safer yet expressive abstraction level by limiting all communication between processes to message\hyp{}passing. CSP also allows to reason about the correctness of the concurrent system, whether certain specifications are met, etc. CSP\hyp{}influenced concurrency has long been proven to serve as a powerful abstraction for creating concurrent systems.

This thesis argues that by combining the parallel nature of CSP with a dynamic multithreaded runtime system, correct and expressive high\hyp{}performance concurrent programs can be created, which are able to fully utilize and scale with the increase of required parallelism in multi\hyp{}core architectures.

The work presented in this thesis is the development of a design and implementation of \Proxc{}: a CSP\hyp{}inspired concurrency library for \Cpp{}. \Proxc{} uses dynamic multithreading to effectively utilize all available logical cores on multi\hyp{}core architectures. The dynamic multithreading is implemented by having a number of runtime schedulers, equal to the available logical processor cores, each running on its own kernel\hyp{}thread. The runtime schedulers employs work stealing for load balancing work dynamically. 

\Proxc{} and other CSP\hyp{}influenced concurrency libraries with dynamic multithreading were benchmarked. The results from the benchmarks showed \Proxc{} to be quite performant compared to the well\hyp{}established programming language Go. However, some results highlighted underlying issues with the runtime system, especially how the current work stealing scheme for the runtime schedulers were flawed in certain situations.

The work done on \Proxc{} should prove useful for designing and implementing runtime systems for dynamic multithreaded CSP\hyp{}based concurrency frameworks. Especially, the design and implementation of the runtime scheduler could in itself be useful for other dynamic multithreaded concurrency frameworks, and not necessarily only for CSP\hyp{}based concurrency frameworks.

A set of potential candidates for future work was identified and outlined for \Proxc{}. Improving the runtime scheduler and the work stealing efficiency is probably the most sought after improvement, which would only increase the multi\hyp{}core performance of the library. Another sought out feature is asynchronous IO operations, which would open up implementing support for networking.

In conclusion, CSP has been proven to provide powerful abstractions for which can be exploited by multi\hyp{}core processors. There is no denying dynamic multithreaded frameworks are the future of high\hyp{}performance systems on multi\hyp{}core architectures, and \Proxc{} is one more example showing the potentiality CSP has on such systems.


%%%%%%%%%%%%%%%%%%%%%%%%%%%%%%%%%%%%%%%%%%%%%%%%%%%%%%%%%%%%%%%%%%%%%%%%%%%%%%%%
\section{Availability}


\Proxc{} is available online, open\hyp{}source and free as in beer, on GitHub \citep{pettersen2017proxcgithub}. Any contributions to the library are welcomed. Any inquiries regarding this thesis or the library can contact the author via mail, \href{mailto:edvard.pettersen@gmail.com}{edvard.pettersen@gmail.com}, or through the project page on GitHub.


%%%%%%%%%%%%%%%%%%%%%%%%%%%%%%%%%%%%%%%%%%%%%%%%%%%%%%%%%%%%%%%%%%%%%%%%%%%%%%%%
\section{Acknowledgment}


I would like to thank Øyvind Teig for the ideas and definitions of the egg, repeat and date timers.

