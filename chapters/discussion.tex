% !TEX encoding = UTF-8 Unicode
%!TEX root = main.tex
% !TEX spellcheck = en-US
%%=========================================


%%%%%%%%%%%%%%%%%%%%%%%%%%%%%%%%%%%%%%%%%%%%%%%%%%%%%%%%%%%%%%%%%%%%%%%%%%%%%%%%%%%%%%%%%%
\chapter{\Proxc{} vs. ProXC}
\label{ch:proxc++_vs_proxc}
% single-core vs multi-core
% performance?
% ergonomics?
% feature difference, why some was chosen or not
% why some features?

As \Proxc{} is a continuation of the project ProXC \citep{pettersen2016proxc}, it is interesting to see the different design choices and capabilities the two projects exhibits.

This chapter compares and discusses the differences between the two projects \Proxc{} and ProXC, which includes differences such as abstractions, library features, usability, and performance. This comparison should provide an insightful look into both libraries, as both cater to the same motivation; providing a CSP abstraction for a programming language with no native support for CSP\hyp{}based concurrency.


%%%%%%%%%%%%%%%%%%%%%%%%%%%%%%%%%%%%%%%%%%%%%%%%%%%%%%%%%%%%%%%%%%%%%%%%%%%%%%%%
\section{Similarities and Differences}


\Cref{tab:library_features_comparison} gives a rough comparison between \Proxc{} and ProXC. The main difference to take from the comparison is the more dynamic feature set \Proxc{} provides compared to the feature set of ProXC. In \Proxc{}, a dynamic number of processes can be spawned in parallel, a dynamic number of alternatives can be alted on, etc. This dynamic approach allows to create more flexible and less hard coded programs, and consequently creating more maintainable and expressive concurrent systems.

\begin{table}
    \centering
    \begin{tabular}{l|l|l}
    
        \ignore{empty} & 
        \multicolumn{1}{c|}{\textbf{\Proxc{}}} & 
        \multicolumn{1}{c}{\textbf{ProXC}} 
        
        \\ \hline
        
        \textbf{\begin{tabular}[c]{@{}l@{}}Target\\ langauge\end{tabular}} & 
        \begin{tabular}[c]{@{}l@{}}The \Cpp{} programming\\ language.\end{tabular} &
        \begin{tabular}[c]{@{}l@{}}The C programming\\ language.\end{tabular}
        
        \\ \hline
        
        \textbf{\begin{tabular}[c]{@{}l@{}}Lightweight\\ processes\end{tabular}} & 
        \begin{tabular}[c]{@{}l@{}}Third party library, portable,\\ customizable stack types.\end{tabular} &
        \begin{tabular}[c]{@{}l@{}}Handwritten, not portable,\\ hard coded stack types.\end{tabular}
        
        \\ \hline
        
        \textbf{\begin{tabular}[c]{@{}l@{}}Process\\ spawning\end{tabular}} &
        \begin{tabular}[c]{@{}l@{}}Synchronous parallel \\ spawning of dynamic\\ number of processes.\end{tabular} &
        \begin{tabular}[c]{@{}l@{}}Synchronous/asynchronous\\ nested parallel and sequential\\ process spawning of\\ fixed number of processes.\end{tabular}
        
        \\ \hline
        
        \textbf{Channels} &
        \begin{tabular}[c]{@{}l@{}}Synchronous and unbuffered,\\ unidirectional, one-to-one, \\ type safe.\end{tabular} &
        \begin{tabular}[c]{@{}l@{}}Synchronous and unbuffered,\\ bidirectional, any-to-any,\\ size safe.\end{tabular}
        
        \\ \hline
        
        \textbf{Alternation} &
        \begin{tabular}[c]{@{}l@{}}Alting on dynamic number \\ of alternatives. Alternatives\\ include channel send and\\ receive, timeouts and skip.\end{tabular} &
        \begin{tabular}[c]{@{}l@{}}Alting on fixed number\\ of alternatives. Alternatives\\ include channel receive,\\ timeouts and skip.\end{tabular}
        
        \\ \hline
        
        \textbf{\begin{tabular}[c]{@{}l@{}}Threading\\ model\end{tabular}} &
        \begin{tabular}[c]{@{}l@{}}M:N, hybrid threading model.\\ Support for multi\hyp{}core.\end{tabular} &
        \begin{tabular}[c]{@{}l@{}}M:1, user threading model.\\ No support for multi\hyp{}core.\end{tabular}                                                      
    \end{tabular}
    \caption{Comparison of library specification and features between \Proxc{} and ProXC}
    \label{tab:library_features_comparison}
\end{table}


%%%%%%%%%%%%%%%%%%%%%%%%%%%%%%%%%%%%%%%%%%%%%%%%%%%%%%%%%%%%%%%%%%%%%%%%%%%%%%%%
\section{Various Capabilities}


A big difference between the two is the threading model used, where \Proxc{} uses a hybrid threading model while ProXC uses a user threading model. The use of threading models only affects the performance in concurrent programs. The same concurrent program running on both \Proxc{} and ProXC should behave just the same. However, since the main development in processor architectures is in multi\hyp{}core, there is an incentive to use the available resources for a potential performance gain.

Now, not everything is about performance. Some might argue correct and expressive abstractions are more important than performance. The abstractions available in \Proxc{} and ProXC are both based on CSP. However, \Proxc{} has a greater expressive power than ProXC because of replicators for parallel spawning and alternatives for alting. Due to this limitation, ProXC cannot express a dynamic numbers of processes and alternatives for alting.


%%%%%%%%%%%%%%%%%%%%%%%%%%%%%%%%%%%%%%%%%%%%%%%%%%%%%%%%%%%%%%%%%%%%%%%%%%%%%%%%
\chapter{Challenges with a Multi\hyp{}Core Library}
\label{ch:difficulty_multicore_csp}
% is performance important?
% using spinlocks vs other locks? hybrids?
% locking kernel-threads to cores

Creating a dynamic multithreaded CSP library for multi\hyp{}core architectures is mostly motivated by the potential performance increase in concurrent systems by utilizing the available cores. CSP is about creating a correct and expressive abstraction over concurrent systems rather than performance, but it is tempting to exploit the apparent parallelism in multi\hyp{}core architectures because of the inherently parallel nature of CSP. Therefore, the rhetorical question is as follows: is the difficulty of implementing a dynamic multithreaded CSP library worth it?


%%%%%%%%%%%%%%%%%%%%%%%%%%%%%%%%%%%%%%%%%%%%%%%%%%%%%%%%%%%%%%%%%%%%%%%%%%%%%%%%
\section{Critical Sections}


What separates a single\hyp{}core vs. a multi\hyp{}core implementation of a user\hyp{}threaded runtime system is the ability with single\hyp{}core runtime systems to reason and control which states processes are in, when they are running, etc. This reasoning is especially important in critical regions of the runtime system, being able to justify certain states in algorithms based on process states.

For a single\hyp{}core runtime system, any critical regions and side effects can be fully reasoned about. Since only one process runs at any given moment, code running between descheduling points essentially becomes a mutual exclusion. Multi\hyp{}core runtime systems cannot follow the same reasoning, since it has much less control whether a given process is currently running on another processor core not. Critical regions in the runtime system must most likely enforce some sort of mutual exclusion. 


%%%%%%%%%%%%%%%%%%%%%%%%%%%%%%%%%%%%%%%%%%%%%%%%%%%%%%%%%%%%%%%%%%%%%%%%%%%%%%%%
\section{Choice of Mutual Exclusion Locks}


What kind of locks a multi\hyp{}core runtime system uses to enforce mutual exclusion is also important. Different locks and their variations, such as OS mutexes, spinlocks, read\hyp{}write locks, and so forth have a different impact on performance in different situations. For instance, given a lock is held for are short\hyp{}term held, then spinlocks are preferred. For locks with high contention, meaning multiple actors are trying to acquire the lock simultaneously, some type of non\hyp{}linear back off procedure is needed. Either way, choosing what kind of lock is suitable for a given situation requires knowledge of different locks and the situation itself.


%%%%%%%%%%%%%%%%%%%%%%%%%%%%%%%%%%%%%%%%%%%%%%%%%%%%%%%%%%%%%%%%%%%%%%%%%%%%%%%%
\section{Non\hyp{}Blocking vs Mutual Exclusion Design}


Sometimes, it is desirable to design a non\hyp{}blocking design rather than mutual exclusion design. A non\hyp{}blocking design is usually much more demanding to develop than a mutual exclusion design, as it is harder to prove the design is correct. Non\hyp{}blocking design is often preferable over mutual exclusion, as it both scales better with a number of processor cores and has a better throughput, but does have a higher latency per operation wise compared to mutual exclusion.


%%%%%%%%%%%%%%%%%%%%%%%%%%%%%%%%%%%%%%%%%%%%%%%%%%%%%%%%%%%%%%%%%%%%%%%%%%%%%%%%
\section{Pinning Kernel\hyp{}Threads to Processor Cores}


Multi\hyp{}core runtime systems must resort to some sort of schedulers, each running on a kernel\hyp{}thread. A factor to consider is whether to set thread affinity for each kernel\hyp{}thread, i.e. pinning each kernel\hyp{}thread to a different processor core. The idea is to avoid the operating system from migrating kernel\hyp{}threads between processor cores, which causes discrepancies in the runtime system. The counter argument is the operating system can help with load balancing kernel\hyp{}threads, when for instance a scheduler with a high work load runs on a less used processor core.


%%%%%%%%%%%%%%%%%%%%%%%%%%%%%%%%%%%%%%%%%%%%%%%%%%%%%%%%%%%%%%%%%%%%%%%%%%%%%%%%
\chapter{Shortcomings and Limitations}
\label{ch:shortcomings_limitations}


Much of the shortcomings and limitations present in ProXC has been improved upon in \Proxc{}, which includes enforcing correct usage and the portability issues with the user\hyp{}thread implementation. However, some issues are present in the current state of \Proxc{}.


%%%%%%%%%%%%%%%%%%%%%%%%%%%%%%%%%%%%%%%%%%%%%%%%%%%%%%%%%%%%%%%%%%%%%%%%%%%%%%%%
\section{Enforcing Correct Usage}


\Proxc{} goes to great lengths to enforce correct usage by using much of the semantic facilities present in the \Cpp{} programming language. However, all problems existing in \Cpp{} programs, such as null pointer dereferencing and memory leaks, are possible in \Proxc{} programs.  

It is impossible to create a system which always generates and enforces correct program behaviour, and somewhere does the line have to be drawn. For instance, channels in \Proxc{} are one\hyp{}to\hyp{}one, and if multiple processes where to operate on the same channel end simultaneously, the result would be undefined behaviour. Channel ends have therefore been made non\hyp{}copyable to enforce the channel end only having a single owner and user. It is however very easy to bypass this by sending a reference of the channel end when spawning a new process, making multiple users still possible. Having shared memory between processes is also entirely possible, but is highly discouraged.

\Proxc{} is all about providing a safe framework for creating concurrent systems. Creating a complete safe framework is impossible, as the foundation of \Cpp{} is inherently unsafe. Therefore, \Proxc{} only enforces correct usage to some degree, and let the rest be up to the programmer.


%%%%%%%%%%%%%%%%%%%%%%%%%%%%%%%%%%%%%%%%%%%%%%%%%%%%%%%%%%%%%%%%%%%%%%%%%%%%%%%%
\section{Inefficient Load Balancing}


\Proxc{} uses work stealing for load balancing work between schedulers. A scheduler continues to run work as long as it has ready work. However, when a scheduler runs out of work and becomes idle, it tries to steal ready work from other schedulers. How these idle schedulers decide when to steal, how much ready work to steal, and so forth is not optimal nor efficient.

Currently, when a scheduler becomes idle it tries to steal once from a random scheduler. If the steal is successful it resumes the stolen work. However, if the steal fails the scheduler sleeps for $1$ millisecond and tries again when it wakes up. There is no coordination between the schedulers, as a scheduler with lots of ready work has no way to indicate to other schedulers to steal from it. 

This sort of brute\hyp{}force approach to find ready work is of course not desirable, as a concurrent program with few parallel tasks will generate lots of unnecessary wake\hyp{}ups of from idle scheduler trying to find work. This periodical wake\hyp{}up can also cause unnecessary migration of tasks between schedulers when few sequential dependent tasks are running, which was highlighted in the concurrent Mandelbrot test in \cref{subsec:concurrent_mandelbrot_test}.

The unnecessary migration of tasks and wasteful wake\hyp{}ups becomes negligible after an amount of parallel semi\hyp{}independent tasks is surpassed in the program.


%%%%%%%%%%%%%%%%%%%%%%%%%%%%%%%%%%%%%%%%%%%%%%%%%%%%%%%%%%%%%%%%%%%%%%%%%%%%%%%%
\section{Wasteful use of Resources}


The \Proxc{} runtime system has to manage a great lot of resources, including user\hyp{}thread management and inter\hyp{}process communication. Much of the resources used with user\hyp{}thread management are one\hyp{}time use, meaning nothing is reused of stack or control data structures related to the user\hyp{}threads. This one\hyp{}time use creates lots of unnecessary allocations of dynamic memory which could have been reused for later use.

A smarter runtime system could detect these one\hyp{}time use resources and reuse the resource rather than deallocating them. This, of course, requires a much more complex resource management by the runtime system, and could potentially introduce a greater overhead.


%%%%%%%%%%%%%%%%%%%%%%%%%%%%%%%%%%%%%%%%%%%%%%%%%%%%%%%%%%%%%%%%%%%%%%%%%%%%%%%%
\section{Determinism and Real\hyp{}Time Characteristics}


An important selling point of CSP is the ability to reason how deterministic a CSP model is, usually via refining the model against a specification model. The very same attribute, determinism, is also important in real\hyp{}time systems. A real\hyp{}time system is where the time it takes to complete an operation is just as important as the result of the operation, and can have direct consequences of the system functionality.

The question is whether \Proxc{} qualifies to be used for any real\hyp{}time requirements. The abstractions provided by \Proxc{} are very much deterministic, except for alting. Alting uses a pseudo\hyp{}random distribution to choose an alternative when multiple alternatives are ready. The reasoning behind using a pseudo\hyp{}random distribution is it favours fairness over determinism.

Another factor of determinism in \Proxc{}, which is unrelated to CSP, is dynamic multithreading. As the runtime schedulers rely on work stealing for distributing parallel work, the deterministic characteristic of the runtime system is affected. An idle scheduler chooses its victim by random, which consequently makes a process being stolen or not random. If a time critical operation relies on certain processes to be distributed among the schedulers to meet its deadline, it is inherently non\hyp{}deterministic whether the deadline is met or not.

